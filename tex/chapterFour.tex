\chapter{Postponed splitting}

Consider the following rule in separation logic

$$\infer{\Gamma_1, \Gamma_2 \vdash \phi_1 \ast \phi_2}
      {\Gamma_1 \vdash \phi_1 &
       \Gamma_2 \vdash \phi_2}$$

From derivation perspective this rule is perfectly fine, since when one writes a derivation, one knows perfectly well what $\Gamma_1$ and $\Delta_2$ are supposed to be in advance.

However, when the perspective is switched and one seeks to find the proof of $\Gamma \vdash \phi_1 \ast \phi_2$, a problem arises:
It's not immediately clear how to split the context $\Gamma$ into $\Gamma_1, \Gamma_2$, so that both $\Gamma_1 \vdash \phi_1$ and $\Gamma_2 \vdash \phi_2$ are provable.

Since propositions in separation logic are frequently thought as resources, this is an instance of ``resource distribution'' problem.

We seek to automate some parts of this problem drawing inspiration from \citet{Harland_Pym_2003}.

\section{Motivation/examples}

Since separating conjunction is one of the fundamental connectives, the ability to postpone decision wchih resources go where is frequently useful.

Naturally, the usability of this grows with the complexity of a decision which user has to make upfront.
Still, let's consider small illustrative example of where we would want to use it.

Take the following statement:
$(A \wand B) \ast (C \wand D) \ast A \ast C \vdash B \ast D$

For a human it is pretty clear what the derivation should be after a moment's consideration.
But for machine it's not immediately clear how to distribute resources, so instead we can postpone this decision and simply say that resources are distributed disjointly.
$$
\infer{(A \wand B) \ast (C \wand D) \ast A \ast C \vdash B \ast D}
      {(A \wand B)[c_0] \ast (C \wand D)[c_1] \ast A[c_2] \ast C[c_3] \vdash B
       &
       (A \wand B)[\neg c_0] \ast (C \wand D)[\neg c_1] \ast A[\neg c_2] \ast C[\neg c_3] \vdash D}
$$

Then we utilize wand application on the left-hand side in both sequents.

Let's take a closer look at the left one:

$$
\infer{(A \wand B)[c_0] \ast (C \wand D)[c_1] \ast
        A[c_2] \ast C[c_3] \vdash B}
      {(C \wand D)[c'_1 \& c_1] \ast
       A[c'_2 \& c_2] \ast C[c'_3 \& c_3] \vdash A
       &
       B[c_0] \ast (C \wand D)[\neg c'_1 \& c_1] \ast
       A[\neg c'_1 \& c_2] \ast C[\neg c'_3 \& c_3] \vdash B}
$$

From the left  we can say that $c'_2 \& c_2$ has to be $\true$, hence both $c'_2$ and $c_2$ must be $\true$.
Moreover, $c'_1 \& c_1$ and $c'_3 \& c_3$ must be $\false$, but on its own it's not saying much about the values.

If we take a look at the right sequent, we can also unify $c_0$ with $\true$ and
$\neg c'_1 \& c_1$, $\neg c'_3 \& c_3$ with $\false$.
And from the left sequent we know that $\neg c'_1 \& c_2$ is $\false$, since the first conjunct is $\false$.

The resulting system of equations allows us to conclude that $c_1$ and $c_3$ are false and values of $c'_1$ and $c'_3$ are arbitrary.

Thus, we have a total constraint assignment, which means as soon as we're done with the left derivation it's clear what resources should go to the right sequent.

\section{Rules for environments with constraints}

$e$ is a constraint.
$V$ is a vector of constraints
\todo{notation explanation here}

Let's start with just rules for separation logic in the same style as presented in \citet{Harland_Pym_2003}

\subsection{Axiom}

$$\infer{\phi[e], \Delta \vdash \phi}
      {e = \true &
       \forall e \in \expr(\Delta). e = 0}$$

This rule correspond to the following axiom in the usual setting:
$$\infer{\phi \vdash \phi}{}$$

\subsection{Separating conjunction}

The crucial rule here is one for separating conjunction:\\
Instead of the usual rule
$$\infer{\Gamma_1, \Gamma_2 \vdash \phi_1 \ast \phi_2}
      {\Gamma_1 \vdash \phi_1 &
       \Gamma_2 \vdash \phi_2}$$

we write the following one:
$$\infer{\Gamma \vdash \phi_1 \ast \phi_2}
      {\Gamma[V] \vdash \phi_1 &
       \Gamma[\overline{V}] \vdash \phi_2}$$

where $\overline{V}$ is an element-wise negated vector of constraints $V$, so
that elements which appear in $\Gamma[V]$ are guaranteed not to be in $\Gamma[\overline{V}]$.

\subsection{Separating implication}

For separating implication the rule practically stays the same, though.

For regular BI logic the rule looks as follows:
$$
\infer{\Gamma \vdash \phi \wand \psi}
      {\Gamma , \phi \vdash \psi}
$$

And with constraints introduced:
$$
\infer{\Gamma \vdash \phi \wand \psi}
      {\Gamma , \phi[\true] \vdash \psi}
$$

Of course, morally it's the same rule since a resource being in the context with constraint $\true$ is precisely the resource being in the context in the usual setting.

While introduction created new constraints, albeit simple ones, elimination forces their resolution.

$$
\infer[ c = \true ]
      {\Gamma, (\phi \wand \psi)[c] \vdash \rho}
      {\Gamma[V] \vdash \phi &
       \Gamma[\overline{V}], \psi[c] \vdash \rho}
$$

Morally, this rules is saying ``in order to use a wand which might be in the context, ensure that it is indeed there and provide resources it needs''.

\subsection{Conjunction}

The only other primitive rule that creates new constraints is (non-separating) conjunction elimination.
This is also where we differ from \citet{Harland_Pym_2003}

Without constraints the rule is saying the following:

$$
\infer[ i \in \{1,2\} ]
      {\Delta, \phi_1 \wedge \phi_2 \vdash \psi}
      {\Delta, \phi_i \vdash \psi}
$$

Essentially forcing the prover to commit to one of the conjuncts.

With them we can not only postopone the choice, but also (unlinke in \cite{Harland_Pym_2003}) manipulate resources, without forcing them to be present.

$$
\infer{\Delta, (\phi_1 \wedge \phi_2)[c] \vdash \psi}
      {\Delta, \phi_1[c' \& c], \phi_1[\neg c' \& c] \vdash \psi}
$$

\subsection{Existential quantifier}

We aslo encounter first major design decision:\\
Given an exitential with constraints in the context, destruction leaks the element from the proof into the context.
Which might lead to problems, if the element allows us to derive something on its own.

For example, one can't simply destruct $(\exists (p : \bot), P x)[c]$, to get a proof of $\bot$ in the ambient logic and $(P x)[c]$ in BI\@.
This is because from this would allow using \emph{ex falso} rule no matter what the constraint evaluates to.

\citet{Harland_Pym_2003} don't encounter this problem, since they take the idea of principal formuals as their guiding principle: ``the principal formula of each rule must be assigned the value of 1'' (we use $\true$ instead of 1).

We also make use of the mantra in this instance for simplification:

$$
\infer[c = \true]
      {\Delta, (\exists x : X, P x)[c] \vdash \phi}
      {x : X &
       \Delta, (P x)[c] \vdash \phi}
$$

However, there are several options, which we will discuss later~\ref{sec:design_decisions_existential}.

\subsection{Rules for IPM}

\begin{itemize}

\item Affinity
  $$
  \infer{\Affine{[]}}{}
  $$

  \begin{equation*}
  \infer{\Affine{ECons\ \Gamma\ (i,\_)\ P}}
        {\Affine{\Gamma} &
         \Affine{P}
       }
  \quad
  \infer{\Affine{ECons\ \Gamma\ (i,\, \false)\ P}}
        {\Affine{\Gamma}}
  \end{equation*}
\item Context manipulation
  \begin{itemize}
  \item iRename
  \item iClear
    $$
    \infer{\Gamma ; P[\true], \Delta \Vdash Q}
          {\Gamma ; \Delta \Vdash Q &
           \Affine{P} \vee \Absorbing{Q}}
    $$
    $$
    \infer{\Gamma ; P[\false], \Delta \Vdash Q}
          {\Gamma ; \Delta \Vdash Q}
    $$
  \item iEval
    $$
    \infer{\Gamma ; \Delta \Vdash Q}
          {\Gamma ; \Delta \Vdash Q' &
           Q' \vdash Q
          }
    $$
  \end{itemize}
\item Assumptions
  $$
  \infer{\Gamma; \Delta \Vdash P}
        {P[\true] \in \Gamma &
         \Absorbing{P} \vee \Affine{\Delta}}
  $$

 $$
  \infer{\Gamma; \Delta \Vdash P}
        {P[\true] \in \Delta &
         \Absorbing{P} \vee \Affine{\Delta \backslash P}}
 $$

  Ex falso
  $$
  \infer{\Gamma; \Delta \Vdash P}
        {\Gamma; \Delta \Vdash \bot}
  $$
\item Intuitionistic/Spatail/Pure transitions
  \begin{itemize}
  \item iIntuitionistic
   %$$
   % \infer{\Gamma ; \Delta \Vdash R}
   %       {P[c] \in \Gamma &
   %        \pers P \vdash \pers Q^{(\text{IntoPersistent}\ \true\ P\ Q)} &
   %        \Gamma [P / Q]; \Delta \Vdash R
   %      }
   %$$
    $$
    \infer{\Gamma ; \Delta \Vdash R}
          {P[c] \in \Delta &
           P \vdash \pers Q^{(\text{IntoPersistent}\ \false\ P\ Q)} &
           \Gamma, Q; \Delta \backslash P \Vdash R
         }
    $$
  \item iSpatial
    $$
    \infer{\Gamma ; \Delta \Vdash R}
          {P[c] \in \Gamma &
           \affine P \vdash Q^{(\text{FromAffinely}\ P\ Q)} &
           \Gamma \backslash P; \Delta, Q \Vdash R
         }
    $$
  \item iPure
    $$
    \infer{\Gamma ; \Delta \Vdash R}
          {\pure{\phi}[\true] \in \Gamma &
           \phi \imp \Gamma \backslash (\pure{\phi}); \Delta \Vdash R
         }
    $$
    $$
    \infer{\Gamma ; \Delta \Vdash R}
          {\pure{\phi}[\true] \in \Delta &
           \Affine{P} \vee \Absorbing{R} &
           \phi \imp \Gamma; \Delta \backslash (\pure{\phi}) \Vdash R
         }
    $$
  \item iEmpIntro
    $$
    \infer{\Gamma ; \Delta \Vdash \emp}
          {\Affine{\Delta}}
    $$
  \item iPureIntro
    \begin{equation}
    \infer{\Gamma ; \Delta \Vdash \pure{\phi}}
          {\phi}
    \quad
    \infer{\Gamma ; \Delta \Vdash \affine \pure{\phi}}
          {\phi & &
           \Affine{\Delta}
          }
    \end{equation}
  \end{itemize}
\item iFrame
\item Intro of wand/implication
\item Revert
\item Specialize and Pose
\item Apply
\item Existential
  \begin{itemize}
  \item Intro
  \item Destruct
  \end{itemize}
\item Modalities
\item iDestruct
\item iIntros
\item Induction
\item Löb
\item Assert
\item Rewrite
\end{itemize}

\section{Design implemented}

\section{Possible designs and comparisons, what do we need}

\subsection{Alternatives for destructing existentials}
\label{sec:design_decisions_existential}


\begin{enumerate}
\item for introduced variables

\begin{itemize}
\item proving the type is inhabited
\item guarding the introduced variable with a proof that constraint is true
\item conditional Maybe
\end{itemize}
\item for propositions

\begin{itemize}
\item conditional empty
\item whole environemnts with proofs of constraints that constraint is equal to true quantified
\end{itemize}
\end{enumerate}

\subsection{The need to solve constraints afterwards for modalities with action "clear"}

\subsection{Environments}

\paragraph{Continuation-style environments}

\paragraph{Boolean constraints with existential variables}

\paragraph{Boolean constraints resolved post-factum with equations posed as goals}

\paragraph{Different styles of environemnts?}

\section{Ltac2 features used}

"Reflection on the use of Ltac2"
Mention that Ltac2 was complete for our purposes
