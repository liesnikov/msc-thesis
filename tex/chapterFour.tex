\chapter{Postponed splitting}

Consider the following rule in separation logic\\

\infer{\Gamma_1, \Gamma_2 \vdash \phi_1 \ast \phi_2}
      {\Gamma_1 \vdash \phi_1 &
       \Gamma_2 \vdash \phi_2}

From derivation perspective this rule is perfectly fine, since when one writes a derivation, one knows perfectly well what $\Gamma_1$ and $\Delta_2$ are supposed to be in advance.

However, when the perspective is switched and one seeks to find the proof of $\Gamma \vdash \phi_1 \ast \phi_2$, a problem arises:
It's not immediately clear how to split the context $\Gamma$ into $\Gamma_1, \Gamma_2$, so that both $\Gamma_1 \vdash \phi_1$ and $\Gamma_2 \vdash \phi_2$ are provable.

Since propositions in separation logic are frequently thought as resources, this is an instance of ``resource distribution'' problem.

We seek to automate some parts of this problem drawing inspiration from \cite{Harland_Pym_2003}.

\subsection{Motivation/examples}



\subsection{Rules for environments with constraints}

\subsubsection{Alternatives for destructing existentials}

\begin{enumerate}
\item for introduced variables

\begin{itemize}
\item proving the type is inhabited
\item guarding the introduced variable with a proof that constraint is true
\item conditional Maybe
\end{itemize}
\item for propositions

\begin{itemize}
\item conditional empty
\item whole environemnts with proofs of constraints that constraint is equal to true quantified
\end{itemize}
\end{enumerate}
\subsubsection{The need to solve constraints afterwards for modalities with action "clear"}

\subsection{Design implemented}

\subsection{Possible designs and comparisons, what do we need}

\subsubsection{Continuation-style environments}

\subsubsection{Boolean constraints with existential variables}

\subsubsection{Boolean constraints resolved post-factum with equations posed as goals}

\subsubsection{Different styles of environemnts?}

\subsection{Ltac2 features used}

"Reflection on the use of Ltac2"
Mention that Ltac2 was complete for our puposes
