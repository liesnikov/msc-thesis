\chapter{Conclusion and future work}

\vspace{-2em}
In this thesis we reimplement MoSeL using a new tactic language Ltac2, using the new version of MoSeL to develop several new features.
In particular, we provide a set of new tactics that support postponed decisions in resource distribution, a goal-matching tactical, and then showcase their usage with a simple solver for separation logic.

This thesis pursued two main goals: explore the potential of Ltac2 and provide new tactics that don't require users to commit to a particular distribution of resources upfront.

Throughout our development, Ltac2 proved to be a language with a solid foundation, which suffices for implementing many advanced tactics.
However, despite leaving an overall positive impressions, it still has issues which complicate the development procedure.
These mostly concern user-facing features, like advanced notations, which don't impact Ltac2 on deeper levels, but make the development of tactics with advanced user interfaces complicated to write.
E.g, we couldn't extend Coq intro patterns to include Iris constructs.
That aside, Ltac2 turned out to be powerful enough to both reimplement (a significant part of) MoSeL and introduce a completely new proof mode for environments with Boolean constraints.

The new proof mode, which introduces \coqe{iSplit}, was implemented almost exactly as described in the original paper~\cite{harlandResourceDistributionBooleanConstraints2003}.
This, combined with the new tactical \coqe{iMatch} allowed us to develop a simple first-order separation logic solver, which manages resources automagically.
For both the new and the old proof modes we don't provide a reimplementation of all MoSeL tactics, but just a selection.

While the work we present here can be used immediately, there is still room for improvement:
\paragraph{The ``lost'' evars issue.}
  This issue breaks the abstraction of the tactics to some extent, as the presented goals seemingly don't have anything to do with the presented entailments.
  We hypothesize that this can be solved by a more careful consideration of the usage of the existing tactics and introduction of heuristics based on the goals.
\paragraph{Performance.}
  While the observed performance of tactics seems reasonable, we are yet to do a study evaluating performance of Ltac2 and the newly implemented tactics.
  However, it is obvious that the new MoSeL proof mode could benefit from optimizations, as seen from both larger proofs and, in particular, performance of \coqe{i_solver}.
\paragraph{More powerful solver to introduce an \texttt{iSpecialize} rule.}
  While our custom solving procedure sufficed for our goals, we had to modify the rules for \coqe{iSpecialize} to let the newly introduced Boolean expressions fit the required format.
  This can be solved either via the introduction of an external SAT solver for the constraints, or an implementation of such a solver directly in Ltac2.
\paragraph{User-facing conveniences.}
  The focus of our work was on the proof-of-concept implementation of MoSeL in Ltac2, which didn't include some of the most useful user-facing features of the Ltac1 version of MoSeL.
  In particular, it would still be possible to reimplement intro patterns in the new proof mode, even without proper support from Ltac2, either by limiting the patterns to those allowed by Coq and reusing them for MoSeL, or by following the original implementation and parsing intro patterns from Coq \coqe{string}s.
  We also don't contribute towards error message improvements of MoSeL, even though Ltac2 does provide a good opportunity for this. Any such improvements to future work too.

That being said, we still consider our pursuit to be a successful experiment, paving a way for improvements both in Ltac2 and MoSeL.

%%% Local Variables:
%%% mode: latex
%%% TeX-master: "thesis"
%%% TeX-parse-self: t
%%% TeX-auto-save: t
%%% reftex-cite-format: natbib
%%% reftex-default-bibliography: ("/home/buzzer/my-dir/ed/uni/saar/prjcts/iris/npm/tex/TacticsProofs.bib")
%%% End: