\chapter{Related work}
\label{cha:related-work}

One of the first advances on finding out what tactics can look like for separation logic was done by \citet{appel2006tactics}.
It built on existing work which embedded separation logic in Coq and developed a set of tactics for stepping through atomic operations, assertion operations and some automation tactics, like \coqe{sep_trivial}.
This was followed by a paper by \citet{mccreightPracticalTacticsSeparation2009}.
It presented a comprehensible set of tactics in Coq, applied them to Cminor, a C-like intermediate language for CompCert and used them to verify a garbage collector.
Afterwards, ``Charge!'' framework \cite{bengtsonCharge2012} developed language- and memory-model independent tactic language, also in Coq.
It provided automation for first-order propositions in separation logic and brought tactics closer to native tactic language of Coq.

\subsection{look at IPM/solving constraints related}

\subsection{programmable tactics}

compare with other tactic languages
Mtac2 gives stronger types to tactics, what can you say about the tactics

\subsection{Context management}
\label{sec:context-management}

\begin{itemize}
\item Idris \cite{MultiplicitiesIdris2Documentation}
\item Tag-based approaches \citet{hodasTagFrameSystemResource2002} \citet{lopezImplementingEfficientResource2004}
  \href{http://www-cgi.cs.cmu.edu/afs/cs/user/fp/www/courses/lp/lectures/15-resources.pdf}{pfenning lectures}
  implemented in Lolli \cite{LolliLinearLogic}
\end{itemize}

%%% Local Variables:
%%% mode: latex
%%% TeX-master: "thesis"
%%% TeX-parse-self: t
%%% TeX-auto-save: t
%%% reftex-cite-format: natbib
%%% reftex-default-bibliography: ("/home/buzzer/my-dir/ed/uni/saar/prjcts/iris/npm/tex/TacticsProofs.bib")
%%% End: